\chapter{Schlussfolgerung}

Der Einsatz von Künstlicher Intelligenz in eingebetteten Systemen markiert einen bedeutenden technologischen Fortschritt. Diese Integration ermöglicht es, komplexe Aufgaben autonom zu lösen und Entscheidungen in Echtzeit zu treffen. Beispiele wie der Tesla Autopilot und Siemens' MindSphere zeigen, wie KI die Navigation in komplexen Verkehrssituationen oder die Optimierung von Wartungsprozessen vorantreibt. Auch der Nest Thermostat von Google verdeutlicht, wie KI den Energieverbrauch effizienter gestalten kann.
Trotz der Vorteile stehen eingebettete Systeme vor Herausforderungen wie begrenzter Rechenleistung und Speicher sowie hohen Echtzeit-Anforderungen, insbesondere in sicherheitskritischen Anwendungen. Die Entwicklung robuster Algorithmen ist entscheidend, um Zuverlässigkeit und Sicherheit zu gewährleisten.
Die erfolgreiche Anwendung von KI in Bereichen wie Industrie 4.0, Gesundheitswesen und Überwachungstechnik belegt, dass KI-basierte eingebettete Systeme Prozesse verbessern und neue Möglichkeiten schaffen können. Zukünftige Forschung sollte sich auf effizientere Algorithmen, energiesparende Hardware und schnellere Echtzeit-Datenverarbeitung konzentrieren. Zudem wird die Sicherstellung von Datenschutz und Datensicherheit immer wichtiger.
Zusammenfassend bietet die Integration von KI in eingebettete Systeme großes Potenzial, erfordert jedoch kontinuierliche Forschung, um die bestehenden Herausforderungen zu meistern und die Technologie weiterzuentwickeln.
