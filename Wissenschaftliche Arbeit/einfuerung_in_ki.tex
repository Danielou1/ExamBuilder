
\section{Einführung in künstliche Intelligenz}

Künstliche Intelligenz ist ein Forschungsfeld, das auf der Nachbildung menschlicher kognitiver Fähigkeiten basiert \cite{Amit.2018, Koricanac.2021, Zhang.2023}. Eine geeignete Analogie, um KI zu erklären, ist das menschliche Gehirn. Dieses besteht aus etwa 85 Milliarden Nervenzellen, sogenannten Neuronen, die kontinuierlich elektrische Impulse erzeugen \cite{Amit.2018}. Jedes Neuron bildet zehntausende Verbindungen zu seinen Nachbarzellen \cite{Amit.2018}. Dieses hochkomplexe System ist die Grundlage für Lernprozesse, Schlussfolgerungen und abstraktes Denken.  

Die zentrale Frage lautet: Ist es möglich, ein solches System künstlich nachzubilden? Was genau ist künstliche Intelligenz?  

\subsection{Definition und Kategorien der künstlichen Intelligenz}

Künstliche Intelligenz basiert im Wesentlichen auf Algorithmen, also Computerprogrammen \cite{Amit.2018, Koricanac.2021, Zhang.2023}. Dabei wird KI in drei Hauptkategorien unterteilt:

1. \textbf{Schwache KI}:  
   Schwache KI ist auf spezifische Aufgabenbereiche spezialisiert\cite{Funk.2022}. viele alltägliche Anwendungen, wie Sprachassistenten (z. B. Siri) oder Spamfilter in E-Mails, basieren auf schwacher KI. Diese Systeme sind sehr leistungsfähig in ihrem jeweiligen Bereich, können ihre Erkenntnisse jedoch nicht auf andere Domänen übertragen.

2. \textbf{Starke KI}:  
   Starke KI beschreibt Systeme, die über ähnliche kognitive Fähigkeiten wie der Mensch verfügen \cite{Funk.2022}. Im Gegensatz zur schwachen KI wäre eine starke KI in der Lage, Schlussfolgerungen von einem Bereich auf andere zu übertragen \cite{Funk.2022}.

3. \textbf{Künstliche Superintelligenz}:  
   Die künstliche Superintelligenz beschreibt hypothetische Systeme, die in allen Aspekten menschliche Intelligenz übertreffen könnten \cite{Funk.2022}.

\subsection{Von schwacher KI zu starker KI}

Ein zentraler Forschungsbereich der KI ist die Frage, wie schwache KI-Systeme in starke KI überführt werden können. Ein Ansatz hierbei ist die Nachahmung des menschlichen Lernprozesses durch künstliche neuronale Netze \cite{Funk.2022}.

Künstliche neuronale Netze bestehen aus miteinander verbundenen künstlichen Neuronen, die in mehreren Schichten organisiert sind \cite{Lammel.2023}. Während der Trainingsphase erhält das Netzwerk große Mengen an Daten. Das Netzwerk lernt durch Rückmeldungen, ob ein Bild korrekt erkannt wurde oder nicht. Das Netzwerk lernt durch Rückmeldungen, ob ein Bild korrekt erkannt wurde oder nicht. Auf dieser Basis werden die Verbindungen zwischen den Neuronen angepasst: Verbindungen, die zu korrekten Ergebnissen führen, werden verstärkt, während Verbindungen, die zu fehlerhaften Ergebnissen führen, geschwächt werden \cite{Lammel.2023}.

Nach zahlreichen Trainingszyklen entwickelt sich das Netzwerk zu einem sogenannten intelligenten neuronalen Netzwerk, das sich selbst optimieren kann. Dieser Prozess wird als \emph{Deep Learning} bezeichnet \cite{Lammel.2023}. Deep Learning hat bereits zahlreiche Bereiche revolutioniert, darunter Bildverarbeitung, Spracherkennung und autonome Systeme \cite{Lammel.2023}.
