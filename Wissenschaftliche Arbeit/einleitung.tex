\chapter{Einleitung}
Die zunehmende Integration von Machine Learning in eingebettete Systeme wird durch mehrere Schlüsselfaktoren motiviert. Ein zentraler Aspekt ist die Fähigkeit von Machine Learning, die Leistung und Funktionalität solcher Systeme erheblich zu verbessern. Eingebettete Systeme, die oft in ressourcenbeschränkten Umgebungen arbeiten, profitieren von der Effizienz und Entscheidungsfähigkeit, die Machine Learning bietet. Beispielsweise ermöglichen fortschrittliche Algorithmen und maschinelles Lernen eine präzise Verarbeitung und Analyse von Sensordaten in Echtzeit, was für Anwendungen im Gesundheitswesen, der Automobilindustrie und in industriellen Umgebungen von entscheidender Bedeutung ist \cite{Gembaczka.2019}.\\
Im Gesundheitswesen können eingebettete ML-Systeme eine Echtzeit-Überwachung und -Analyse von Vitaldaten ermöglichen \cite{Gembaczka.2019}, was zu einer verbesserten Patientenüberwachung und -versorgung führt. In der Automobilindustrie unterstützen Machine-Learning-gesteuerte eingebettete Systeme fortschrittliche Fahrerassistenzsysteme (ADAS) und autonomes Fahren, indem sie Umgebungsdaten verarbeiten und schnelle Entscheidungen treffen. Industrie 4.0 profitiert von Machine Learning in eingebetteten Systemen durch die Optimierung von Produktionsprozessen und die vorausschauende Wartung von Maschinen, was zu einer höheren Effizienz und geringeren Ausfallzeiten führt \cite{Gembaczka.2019}.\\
Das Ziel dieser Arbeit besteht darin, einen umfassenden Einblick in die Nutzung von Machine Learning in eingebetteten Systemen zu gewähren. Es werden sowohl die grundlegenden Konzepte als auch deren Anwendungsbereiche beschrieben. Ein besonderer Schwerpunkt liegt auf der Integration von Machine Learning in eingebettete Systeme, wobei die Vorteile, Herausforderungen und praktischen Implementierungsbeispiele analysiert werden. Ziel ist es, ein tieferes Verständnis für die Potenziale und Grenzen von Machine Learning in diesem Kontext zu schaffen und mögliche zukünftige Forschungsrichtungen aufzuzeigen.