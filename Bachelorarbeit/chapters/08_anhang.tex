\chapter*{Anhang}
\addcontentsline{toc}{chapter}{Anhang}
\label{chap:anhang}

% Dieses Kapitel enthält eine vollständig mit der ExamBuilder-Anwendung generierte Klausur.
% Sie dient als praktisches Beispiel, um die vielfältigen Exportfunktionalitäten und
% die Qualität der Ausgabeformate des ExamBuilders zu demonstrieren.

\section{Beispielklausur: Projektmanagement (Generiert mit ExamBuilder)}
\label{anhang:beispielklausur}

% Die nachfolgende Klausur wurde vollständig mit der ExamBuilder-Anwendung generiert.
% Sie dient als reales Beispiel für die Exportfunktionalität und die Ausgabeformate.
\includepdf[pages=-, addtotoc={1,section,1,Generierte Klausur,klausur:generated}]{../Klausur.pdf}

\subsection{Begründung der Repräsentativität}
\label{anhang:begruendung}
Die oben dargestellte Klausur zum Thema "Projektmanagement" (siehe \ref{anhang:beispielklausur}) wurde als repräsentatives Beispiel gewählt, da sie eine vielfältige Kombination von Elementen enthält, die die Kernfunktionalitäten des ExamBuilders demonstrieren und den Nutzen der Anwendung unterstreichen:
\begin{itemize}
\item \textbf{Umfassende Metadaten und Hinweise:} Das Deckblatt der Klausur enthält detaillierte Informationen wie Universität, Fachbereich, Modul und Semester. Darüber hinaus sind spezifische Prüfungshinweise sowie Angaben zur Bearbeitungszeit und den zugelassenen Hilfsmitteln (z.B. Wörterbuch) enthalten. Diese Elemente zeigen die Fähigkeit des ExamBuilders, komplexe Metadaten effizient zu verwalten und in das Exportdokument zu integrieren.
\item \textbf{Vielfältige Fragetypen und Hierarchie:} Die Klausur beinhaltet eine breite Palette an Fragetypen, darunter offene Fragen, Multiple-Choice-Fragen (MCQs), Lückentextaufgaben und Richtig/Falsch-Fragen. Die Fragen sind hierarchisch strukturiert, mit Hauptfragen (z.B. "Projektdefinition und Projektrisiken") und dazugehörigen Unterfragen, was die flexible Fragenverwaltung des ExamBuilders verdeutlicht.
\item \textbf{Detaillierte Punktevergabe:} Jede Frage und Unterfrage ist mit einer spezifischen Punktzahl versehen, und die Gesamtpunktzahl wird korrekt ausgewiesen, was die präzise Punkteverwaltung des ExamBuilders demonstriert.
\item \textbf{Integration von Code-Beispielen:} Die Klausur enthält ein eingebettetes Code-Beispiel (\texttt{int x = 10;}, sowie einen Java-Codeblock). Dies illustriert die Fähigkeit des ExamBuilders, formatierte Code-Segmente in Prüfungsfragen zu integrieren und darzustellen.
\item \textbf{Layout-Steuerung und Lesbarkeit:} Die Klausur nutzt Seitenumbrüche (`Die Aufgabe folgt auf der nächsten Seite bzw. Rückseite.`) und stellt die Studierendeninformationen (Name, Matrikelnummer) sowie die Prüfungsbewertungstabelle übersichtlich dar. Dies hebt die Bedeutung der Layout-Steuerungsfunktionen des ExamBuilders für die Erstellung professioneller Prüfungsdokumente hervor.
\item \textbf{Eingabe von Studierendendaten und Korrekturhilfen:} Das Deckblatt bietet klare Felder für Name, Matrikelnummer und Unterschrift der Studierenden, während ein Abschnitt für Prüfende die Bewertungstabelle mit maximal erreichbaren Punkten pro Aufgabe enthält. Dies zeigt die vollständige Funktionalität des ExamBuilders zur Generierung vollständiger Prüfungsmaterialien.
\end{itemize}
Zusammenfassend demonstriert diese Beispielklausur umfassend die Stärken des ExamBuilders bei der strukturierten Erfassung, Verwaltung und dem Export von komplexen, multimedialen und spezifisch formatierten Prüfungsfragen, was die Effizienz und Qualität der Prüfungserstellung erheblich verbessert.
