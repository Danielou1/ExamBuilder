\chapter*{Anhang}
\addcontentsline{toc}{chapter}{Anhang}
\label{chap:anhang}

\section{Beispielklausur: Grundlagen der Informatik (Generiert mit ExamBuilder)}
\label{anhang:beispielklausur}

% Die nachfolgende Klausur wurde vollständig mit der ExamBuilder-Anwendung generiert.
% Sie dient als reales Beispiel für die Exportfunktionalität und die Ausgabeformate.
\includepdf[pages=-, addtotoc={1,section,1,Generierte Klausur,klausur:generated}]{../Klausur.pdf}

\subsection{Begründung der Repräsentativität}
\label{anhang:begruendung}
Die oben dargestellte Klausur wurde als repräsentatives Beispiel gewählt, da sie eine typische Kombination von Elementen enthält, die bei der Erstellung von Prüfungen im MINT-Bereich häufig vorkommen und deren manuelle Zusammenstellung fehleranfällig und zeitaufwendig ist:
\begin{itemize}
\item \textbf{Hierarchische Komplexität:} Die Klausur enthält Hauptaufgaben ("Übungen"), die in mehrere nummerierte Teilaufgaben und diese wiederum in weitere Unterpunkte (a, b, c) unterteilt sind. Diese Verschachtelung ist typisch für Prüfungen, die ein Thema aus verschiedenen Blickwinkeln beleuchten.
\item \textbf{Vielfältige Inhaltstypen:} Es wird eine Mischung aus reinem Fließtext (für Definitionen und Erklärungen), formatiertem Text (\texttt{Code}, \textbf{Fett}, \textit{Kursiv}) und grafischen Elementen (Zustandsdiagramm) verwendet. Die korrekte Einbettung und Formatierung dieser unterschiedlichen Inhalte ist eine zentrale Herausforderung, die der ExamBuilder adressiert.
\item \textbf{Strukturierte Daten und Listen:} Die Verwendung von nummerierten Listen und verschachtelten Aufzählungen ist ein Standardmittel zur Strukturierung von Fragen und Antwortmöglichkeiten.
\item \textbf{Layout-Steuerung:} Die Anforderung, dass "Übung 2" auf einer neuen Seite beginnt, ist ein gängiger Wunsch von Lehrenden, um die thematische Trennung klar im Layout widerzuspiegeln. Dies demonstriert die Notwendigkeit der manuellen Seitenumbruch-Funktion.
\end{itemize}
Zusammenfassend deckt diese Beispielklausur die wesentlichen Anwendungsfälle ab, für die der ExamBuilder entwickelt wurde: die strukturierte Erfassung, Verwaltung und der Export von hierarchischen, multimedialen und spezifisch formatierten Prüfungsfragen.