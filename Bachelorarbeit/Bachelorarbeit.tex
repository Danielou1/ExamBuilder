% Thesis Title: Konzeption, Design und Implementierung einer Softwarelösung
% zur automatisierten Erstellung und Verwaltung von Prüfungen: Der ExamBuilder
\documentclass[
BCOR=8.25mm,         % Bindekorrektur
DIV=11,              % Satzspiegel
openany,
parskip=half,        % Abstand zwischen Absätzen
bibliography=totoc,  % Literaturverzeichnis im Inhaltsverzeichnis
headsepline=on,      % Trennlinie Kolumnentitel
]{scrbook}

% Préambule
\usepackage[a4paper, top=2cm, bottom=2cm, left=2cm, right=2cm]{geometry}
\usepackage[ngerman]{babel} % Deutsche typogr. Regeln + Trenntabelle
\usepackage[T1]{fontenc}             % TeX Font-Codierung
\usepackage{lmodern}                 % Latin Modern font
\usepackage[utf8]{inputenc}          % Font-Codierung der Eingabedatei
\usepackage[babel]{csquotes}         % Anführungszeichen
\usepackage{graphicx} % Graphiken
\usepackage{here}  % Neu eingefügt
\usepackage{caption}                 % Bildunterschrift
\usepackage{wrapfig}
\usepackage{tabularx}   % Für flexible Spaltenbreiten
\usepackage{booktabs}   % Für \toprule, \midrule, \bottomrule
\usepackage{float}      % Für [H] Platzierung
\usepackage{minted}
% Konfiguration für minted für ein schöneres Aussehen
\setminted{
frame=lines,         % Rahmen um den Code
framesep=2mm,        % Abstand des Rahmens
baselinestretch=1.2, % Zeilenabstand
fontsize=\small,     % Kleinere Schriftgröße
linenos              % Zeilennummern
}
\usepackage{amsmath}	               % Mathematik
\usepackage[pdftex]{hyperref}
\usepackage{xurl} % Erlaubt URL-Umbrüche
\usepackage{pdfpages} % Paket zum Einbinden von PDF-Dateien
\hypersetup{
bookmarksopen=true,
bookmarksopenlevel=3,
colorlinks,
citecolor=blue,
linkcolor=blue,
}
\usepackage{scrhack}  % Unterdrückt Fehlermeldung von listings
\usepackage{microtype}
\usepackage{tabularx}
\usepackage{fancyhdr}
\pagestyle{fancy}

% Clear all headers and footers
\fancyhf{}

% Header: chapitre courant à gauche, rien à droite
\fancyhead[L]{\nouppercase{\leftmark}}   % Chapitre courant
\fancyhead[R]{}

% Footer: logo + nom + page
\fancyfoot[C]{
\includegraphics[width=6cm]{img/mni-logo.pdf}%
\hspace{0.4cm}%
Danielou Mounsande Sandamoun\hspace{0.7cm} \thepage%
}

% Ligne de séparation für header und footer
\renewcommand{\headrulewidth}{0.4pt}
\renewcommand{\footrulewidth}{0.4pt}

% Configuration du Stil für die Bibliographie
\usepackage[
backend=biber,       % Verwende Biber als Backend
backref,             % Rückverweise von Literatur ins Dokument
style=numeric,       % Numerischer Zitierstil
sorting=none,         % Sortierung nach Name, Jahr, Titel
isbn=false           % ISBN in der Bibliographie ausblenden
]{biblatex}

% Lade die Citavi-Bibliothek
\addbibresource{references.bib} % Name der .bib-Datei

%% Numérotation des sections und Tiefe des Inhalts
\setcounter{tocdepth}{3}             % 3 Stufen im Inhaltsverzeichnis
\setcounter{secnumdepth}{3} 	     % 3 Stufen in Abschnittnummerierung

% ----------------------------------------------------------------------------
\begin{document}

%% Titelseite
\begin{titlepage}
\centering
\includegraphics[width=0.9\textwidth]{img/mni-logo.pdf}\[2cm]

\textbf{\huge\sffamily Bachelorarbeit}\[2cm]
\Large \textbf{\textit{Konzeption, Design und Implementierung einer Softwarelösung zur automatisierten Erstellung und Verwaltung von Prüfungen: Der ExamBuilder}}\[1.5cm]
\Large zur Erlangung des akademischen Grades
Bachelor of Science (B.Sc.)
\vspace{1cm}

Vorgelegt dem\[1cm]
Fachbereich Mathematik, Naturwissenschaften und Informatik
der Technischen Hochschule Mittelhessen\[1.5cm]
\textbf{Danielou Mounsande Sandamoun}\[1.3cm]

im Dezember 2025

\vspace{5cm}
\hspace*{-7cm}
\begin{tabular}{ll}
Referent der Arbeit: & \textbf{Prof. Dr. Steffen Vaupel}\[0.5cm]
Korreferent: & \textbf{Prof. Dr. Thorsten Weyer}
\end{tabular}
\vspace{2cm}

\vspace*{\fill}
\end{titlepage}
\cleardoubleemptypage

%% Inhaltsverzeichnis
\tableofcontents

\mainmatter
\pagestyle{fancy}

% --- Kapitel einfügen ---
\input{chapters/01_einleitung.tex}
\input{chapters/02_grundlagen.tex}
\input{chapters/03_anforderungsanalyse.tex}
\input{chapters/04_konzeption_entwurf.tex}
\input{chapters/05_implementierung.tex}
\input{chapters/06_evaluation.tex}
\input{chapters/07_zusammenfassung_ausblick.tex}

\backmatter 

\chapter*{Anhang}
\addcontentsline{toc}{chapter}{Anhang}
\label{chap:anhang}

\section{Beispielklausur: Grundlagen der Informatik (Generiert mit ExamBuilder)}
\label{anhang:beispielklausur}

% Die nachfolgende Klausur wurde vollständig mit der ExamBuilder-Anwendung generiert.
% Sie dient als reales Beispiel für die Exportfunktionalität und die Ausgabeformate.
\includepdf[pages=-, addtotoc={1,section,1,Generierte Klausur,klausur:generated}]{../Klausur.pdf}

\subsection{Begründung der Repräsentativität}
\label{anhang:begruendung}
Die oben dargestellte Klausur wurde als repräsentatives Beispiel gewählt, da sie eine typische Kombination von Elementen enthält, die bei der Erstellung von Prüfungen im MINT-Bereich häufig vorkommen und deren manuelle Zusammenstellung fehleranfällig und zeitaufwendig ist:
\begin{itemize}
\item \textbf{Hierarchische Komplexität:} Die Klausur enthält Hauptaufgaben ("Übungen"), die in mehrere nummerierte Teilaufgaben und diese wiederum in weitere Unterpunkte (a, b, c) unterteilt sind. Diese Verschachtelung ist typisch für Prüfungen, die ein Thema aus verschiedenen Blickwinkeln beleuchten.
\item \textbf{Vielfältige Inhaltstypen:} Es wird eine Mischung aus reinem Fließtext (für Definitionen und Erklärungen), formatiertem Text (\texttt{Code}, \textbf{Fett}, \textit{Kursiv}) und grafischen Elementen (Zustandsdiagramm) verwendet. Die korrekte Einbettung und Formatierung dieser unterschiedlichen Inhalte ist eine zentrale Herausforderung, die der ExamBuilder adressiert.
\item \textbf{Strukturierte Daten und Listen:} Die Verwendung von nummerierten Listen und verschachtelten Aufzählungen ist ein Standardmittel zur Strukturierung von Fragen und Antwortmöglichkeiten.
\item \textbf{Layout-Steuerung:} Die Anforderung, dass "Übung 2" auf einer neuen Seite beginnt, ist ein gängiger Wunsch von Lehrenden, um die thematische Trennung klar im Layout widerzuspiegeln. Dies demonstriert die Notwendigkeit der manuellen Seitenumbruch-Funktion.
\end{itemize}
Zusammenfassend deckt diese Beispielklausur die wesentlichen Anwendungsfälle ab, für die der ExamBuilder entwickelt wurde: die strukturierte Erfassung, Verwaltung und der Export von hierarchischen, multimedialen und spezifisch formatierten Prüfungsfragen. % Einbinden des Anhangs

%% Abbildungsverzeichnis
\listoffigures

%% Literatur
\printbibliography

\end{document}